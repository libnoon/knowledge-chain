% latex document.

\documentclass{article}

\title{The knowledge chain}
\author{Fabrice {\sc Bauzac}}

\begin{document}

\maketitle

\section{About this document}

This document is a draft of an idea I had about how to keep human
knowledge.  Books alone are not enough, because they do not change and
are not worked on whenever a new discovery or precision becomes known.
Also, they will not answer questions or receive and respond to
critics.

\section{The draft itself}

Let $G_s$ be the group of people responsible for maintaining the subject $s$.  Each person in this group is able to convince anyone knowing the prerequisites $P_s$ of the validity of the knowledge $s$.

The people of $G_s$:

\begin{itemize}

\item convince each other so that there is no stray errors, farfelue
  ideas, derives, and make sure everything is ``{\bf credible}'';

\item make a summary: the {\bf results} which people like you and me
  can count on and build from there.

\end{itemize}

Let $\rightarrow$ mean ``allows to learn'':

$$G_{base} \rightarrow 
\left\lbrace
\begin{array}{c}
G_{Fizeau's experiment} \\
G_{The Set theory} \rightarrow G_{Maths} \\
\end{array}
\right\rbrace
\rightarrow G_{Results of Fizeau's experiment}$$

The goal of this is to produce {\bf trustable} summaries for {\bf
  anyone}, and {\bf experts} in case of problem on any point.

\end{document}
